\chapter{Introduction}
Pressure ulcers (PU) have been defined as an area of localised damage to the skin
and underlying tissue caused by pressure, shear, friction or a combination of these
factors (European Pressure Ulcer Advisory Panel). Pressure ulcers are not only costly affecting a significant proportion of the population, but are associated with increased mortality.\\

According to the staging description of PU revised by International NPUAP- EPUAP Pressure Ulcer, in pressure ulcer stage I, Skin is not broken but is red or discolored or may show changes in hardness or temperature compared to surrounding areas. When you press on it, it stays red and does not lighten or turn white. The redness or change in color does not fade within 30 minutes after pressure is removed. And most importantly, A pressure sore at this stage can be reversed in about three days if all pressure is taken off the site.\\

They occur when unrelieved pressure, typically over a bony protrusion, initially causes blood to pool at the site. Blood vessels are constricted and the skin is starved of oxygen and other nutrients. If not diagnosed and relieved of pressure early, the skin or even underlying tissue and bone may ulcerate. With an aging population, incidence is
likely to increase.\\

Early detection is thus key to avoiding pressure ulcers and the subsequent treatment costs, with several non-invasive techniques, including image processing and analysis, recently being developed for this purpose.\\

A study conducted by the U.S. Centers for Medicare and Medicaid Services (formerly Health Care Financing Administration) reported that the proportion of black and Hispanic patients developing more serious Stage III and IV pressure ulcers was over twice that for white
patients \cite{Lyder}. This suggests difficulty in early detection of pressure ulcers in patients with darkly skins, which requires more accurate and efficient image process algorithms to enhance the image distinguished by the medical staff.\\

The aim of this project is to explore the feasibility of using non-invasive techniques for the early detection of pressure ulcers. Emphasis will be placed on exploring a range of image processing and image analysis techniques and develop an practical non-invasive image-based pressure ulcer detection software.