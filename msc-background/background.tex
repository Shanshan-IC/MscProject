\chapter{Background}
\section{Pressure Ulcers}
\indent A pressure ulcer (PU) is any lesion caused by unrelieved pressure resulting in damage of underlying tissue \cite{Wikipedia}. Pressure ulcers are areas of localized tissue destruction caused by compression of soft tissue between a bony prominence and an external surface for a prolonged period of time \cite{Treatment}. Unrelieved pressure and shear are the most important contributing factors in the pathogenesis.\\
\indent In UK, nearly 700,000 people are affected by pressure ulcers each year, across all care settings, including patients in their own homes, with the most vulnerable of patients aged over 75. Around 186,617 patients develop a pressure ulcer in hospital each year \cite{NHS}. Given the aging of the population, PU are likely to become a problem of increasing proportion in the near future \cite{Maklebust}.\\
\indent Pressure Ulcers are associated with significant economic burden, and costs continue to rise \cite{Allman}. The cost of treating a pressure ulcer varies from \pounds 1,064 to \pounds 10,551. Costs increase with ulcer grade because the time to heal is longer and because the incidence of complications is higher in more severe cases. The total cost in the UK is \pounds 1.4 \- \pounds 2.1 billion annually (4\% of total NHS expenditure) \cite{Bennett}. Pressure ulcers cost \textdollar 9.1 -\textdollar 11.6 billion per year with 2.5 million patients in the US. Cost of individual patient care ranges from \textdollar 20,900 to \textdollar 151,700 per pressure ulcer \cite{HumanService}. Pressure ulcers are not only costly, but are associated with increased mortality.\\
\indent Therefore, these consequences highlight the value of the early detection of pressure ulcers. If pressure ulcers are detected in the early stage, simple care of the tissue can allow the skin to remain intact and heal without scarring or the need for surgical intervention. \\
\indent According to the staging description of PU revised by International NPUAP- EPUAP Pressure Ulcer \cite{Epuap}, stage I PU presents as Intact skin with non-blanchable erythema of a localized area usually over a bony prominence. Discoloration of the skin, warmth, edema, hardness or pain may also be present. Darkly pigmented skin may not have visible blanching. And Stage I may be difficult to detect in individuals with dark skin tones due to increased melanin content. 

\section{Related Research}
Main issues regarding proper characterization of skin lesions consists of image acquisition, the image processing and analysis, the feature extraction, and the classification methodology \cite{Patel}. Major advantage of using computer is that patients do not have to undergo many painful diagnosing techniques. Moreover it speeds up the procedure of diagnosis of the disease according to the processed images of skin. There are several previous studies to show what techniques are used to improve the early detection of pressure ulcers.\\
\indent Image processing and computational techniques have been applied in different aspects of wound diagnosis. Some aspects involve wound-area identification, which has been tackled with different techniques such as contour detection with
histogram segmentation, active contours modelling, region growing, clustering approaches or skin texture models \cite{Cula} \cite{Jones} \cite{Karkanis}. Some approaches focus on
detecting the different tissues existing in the wound, by using diverse segmentation methods—such as histogram thresholding, watersheds, mean-shift smoothing, region growing, classification or graphs—sometimes combined with machine learning strategies \cite{Kosmopoulos} \cite{Veredas} \cite{Wannous} \cite{Treuillet}.\\
\indent Erythema can be detected with relatively simple and low-cost imaging and image enhancement techniques at pressure ulcers stage I, which was suggested by Prabhu Jude Rajendran. They used CLAHE algorithm and the preliminary results clearly indicate that the enhanced images exhibit higher contrast and make the pressure ulcer site more conspicuous to the examiner \cite{Rajendran}.\\
\indent Jon Leachtenauer research group used a non-contact imaging-based method to detect Stage I pressure ulcers over a wide range of melanin levels. Two approaches were explored: the first used broad and narrow band visible spectrum imaging, and the second used near infrared (NIR) imaging. The results are presented together with results of numerical analysis of different erythema indices derived from the visible spectrum images \cite{Leachtenauer}. \\
\indent A study at the University Hospital of University of Sao Paulo revealed a pressure ulcers incidence of 39.8\% \cite{Rogenski}, and the evolution of healing process is followed by measuring PU using simple measures, tracing, photographs or image analysis by a computational system, which is sup ported by softwares Motic and AutoCAD \cite{minister}.\\
\indent R.Guadagnin, R. de S. Neves et al initially captured images by digital camera. Each image was separated in the images that correspond to blue, green and red bands. They then performed convolution on each image with a 9x9 mean mask with Idrisi Software to remove distortion due to texture peculiarities as it may happen that due to texture peculiarities isolated points distort parts of the images that belong to healthy skin, pressure border or pressure core \cite{Santana}. \\
\indent Hairong Qi et al. describe a custom geometric correction method to restore the image from the misalignment distortion and present a binary tree-based generic demosaicking algorithm to efficiently estimate the missing special components
and reconstruct a high-resolution full-spectral image. They used this new method for early detection of pressure ulcers, particularly so for dark pigmented skin to show how the geometric correction and demosaicking algorithms successfully reconstruct a full-spectral image from which apparent contrast enhancement between damaged skin and the normal skin is observed \cite{Qi}.\\
\indent Jean-François Deprez proposed a 2D ultrasound elastography based on ultrasound imaging to address the challenging problem of pressure ulcer early detection. This was able to detect a subepidermal layer, helping to differentiate between pathological and healthy regions \cite{Deprez}. \\