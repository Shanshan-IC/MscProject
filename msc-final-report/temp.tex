
Objective measures must be established to correlate the degree of color change to the degree of arterial or venous occlusion. Also, blood pressure changes and how they affect free flap perfusion and color change on EVM must be assessed. Despite the lack of standard values, EVM can still be used to visualize changes in perfusion if an initial video is taken after anastomosis and inset to use as a baseline. In this way, EVM may serve as an adjunct to other well-established methods of flap monitoring.
We found that filming of flaps after dissection of the flap to be transferred, but before ligation of the pedicle from the donor site, produced inferior results compared to filming after anastomosis to the receiving site in the head and neck. We hypothesize that this is due to greater perfusion of the head and neck region, such that blood flow through the face and mucosal surfaces is more apparent and can be more easily contrasted to the free flap blood flow. Another possibility is that the feeding artery and receiving vein within the head and neck provide better perfusion to the free flap than its native vessels. Furthermore, issues with reperfusion hyperemia cannot be discounted.
A current impediment to using EVM as a free flap monitoring tool is the need for the subject and the video camera to be relatively motionless, because motion artifact can be confused with changes in color indicating perfusion alterations. However, software adaptations may be able to filter out nonpatterned motion (as opposed to heartbeats) to diminish or eliminate motion artifact. Also, adequate lighting of the free flap must be available such that video recordings are able to detect changes in color. Research is needed to establish standard parameters for and to verify the accuracy and reliability of EVM. Nevertheless, this initial proof of concept study may serve as an impetus for further research by highlighting the potential of EVM as a noninvasive, sensitive, cost-effective, easy-to-use, and reproducible flap monitoring tool.


Both the variations in colors and movements cause temporal changes of the pixel values, and they are difficult to distinguish from one another as long as they both occur in the emphasized frequency range. If the camera or the recorded object are not kept completely still, the slightest movements can easily cause strong changes in colors as well as a perception of strongly emphasized movements in the EVM magnified video. Typical muscle shakes are significantly correlated with the persons pulse. This fact along with the way the Smartphones are typically used, pose a serious issue for developing a fingerprint liveness detection solution for Smartphones by means of the EVM approach.\\







