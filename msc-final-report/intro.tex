\chapter{Introduction}
Pressure ulcers (PU) have been defined as an area of localized injuries to the skin or underlying tissue that were usually caused by pressure, or pressure in combination with friction or shear.
and underlying tissue caused by pressure, shear, friction or a combination of these
factors (European Pressure Ulcer Advisory Panel) \cite{epuap}. Pressure ulcers are not only significantly affecting increased health costs, but are associated with increased mortality.\\

According to the staging description of PU revised by International NPUAP- EPUAP Pressure Ulcer, in pressure ulcer stage I, skins are not broken but are discolored or red or may show changes in temperature or hardness compared to surrounding areas. It stays red and does not lighten or turn white when pressing it. The changes in color does not fade within 30 minutes after pressures are removed. And most importantly, A pressure sore at this stage can be reversed in about three days if all the pressures are taken off the site \cite{msktc}.\\

They occur when unrelieved pressure, typically over a bony protrusion, initially causes blood to pool at the site. Blood vessels are constricted and the skin is starved of oxygen and other nutrients. If not diagnosed and relieved of pressure early, the skin or even underlying tissue and bone may ulcerate. With an aging population, incidence is
likely to increase.\\

Early detection is thus the key to avoiding pressure ulcers and the subsequent treatment costs, with several non-invasive techniques, including image processing and analysis, recently being developed for this purpose.\\

A study conducted by the U.S. Centers for Medicare and Medicaid Services reported that compare to the white patients, Hispanic and black patients developed more serious Stage III and IV pressure ulcers \cite{Lyder}. This suggests difficulty in early detection of pressure ulcers in patients with darkly skins, which requires more accurate and efficient image process algorithms to enhance the image distinguished by the medical staff.\\

The aim of this project is to explore the feasibility of using non-invasive techniques for the early detection of pressure ulcers. Emphasis will be placed on exploring a range of image processing and image analysis techniques and develop an practical non-invasive image-based pressure ulcer detection software.

\section{Aims and Objectives}
The aims and objectives of this study are:
\begin{itemize}
    \item Exploring a range of image and video processing techniques.
    \item Implement the image analysis method and eulerian color magnification to the software and detect the disease area automatically.
    \item Software development is done using C/C++ and a suitable image processing library. 
    \item Exploring the feasibilities of mobile application for this project.
\end{itemize}
\section{Contributions}
This study extends the work of Mohammed Umar Riyaz \cite{Mohammed} who confirmed that image processing can be used to improve image contrast and subsequently aid in the detection of erythema. The image processing techniques elucidated in his paper can be
adapted to the characteristics of the patients. In the meantime, it was the first time to apply eulerian colour magnification to detect skin pathology. \\

For Eulerian Video Magnification, previous related work is done through the open-source software developed by Fredo Durand \cite{Fredo} from the Massachusetts Institute of
Technology. 


\section{Report Outline}
\begin{itemize}
    \item Chapter 2 Background: Present some background on image analysis applied in detection for medical use. It also looks at previous work that has been done. 
    \item Chapter 3 Specification: Outlines the requirements of the each component of the software, broken down into small tasks to handle.
    \item Chapter 4 Project Design: Present the details of the software prototype.
    \item Chapter 5 Image Process: Present the details of related algorithms used in the report and how to achieve it.
    \item Chapter 6 Experimental Results of Eulerian Color Magnification: Presents a brief description of the training and testing dataset for Eulerian Color Magnification.
    \item Chapter 7 Methodology: Development technique adopted by the software and issues faced during the development process.
    \item Chapter 8 Conclusion and Future Work:  An overall discussion of the results of the image analysis and video process in the previous chapter, an outline of some limitations in the current system and exploring the possibilities of future research.
\end{itemize}





